\documentclass[journal,12pt,twocolumn]{IEEEtran}

\usepackage{setspace}
\usepackage{gensymb}
\singlespacing
\usepackage[cmex10]{amsmath}
\usepackage{amssymb}
\usepackage{xurl}

\usepackage{amsthm}
\usepackage{comment}
\usepackage{mathrsfs}
\usepackage{txfonts}
\usepackage{stfloats}
\usepackage{bm}
\usepackage{cite}
\usepackage{cases}
\usepackage{subfig}

\usepackage{longtable}
\usepackage{multirow}

\usepackage{enumitem}
\usepackage{mathtools}
\usepackage{steinmetz}
\usepackage{tikz}
\usepackage{circuitikz}
\usepackage{verbatim}
\usepackage{tfrupee}
\usepackage[breaklinks=true]{hyperref}
\usepackage{graphicx}
\usepackage{tkz-euclide}

\usetikzlibrary{calc,math}
\usepackage{listings}
    \usepackage{color}                                            %%
    \usepackage{array}                                            %%
    \usepackage{longtable}                                        %%
    \usepackage{calc}                                             %%
    \usepackage{multirow}                                         %%
    \usepackage{hhline}                                           %%
    \usepackage{ifthen}                                           %%
    \usepackage{lscape}     
\usepackage{multicol}
\usepackage{chngcntr}

\DeclareMathOperator*{\Res}{Res}

\renewcommand\thesection{\arabic{section}}
\renewcommand\thesubsection{\thesection.\arabic{subsection}}
\renewcommand\thesubsubsection{\thesubsection.\arabic{subsubsection}}

\renewcommand\thesectiondis{\arabic{section}}
\renewcommand\thesubsectiondis{\thesectiondis.\arabic{subsection}}
\renewcommand\thesubsubsectiondis{\thesubsectiondis.\arabic{subsubsection}}


\hyphenation{op-tical net-works semi-conduc-tor}
\def\inputGnumericTable{}                                 %%

\lstset{
%language=C,
frame=single, 
breaklines=true,
columns=fullflexible
}
\begin{document}


\newtheorem{theorem}{Theorem}[section]
\newtheorem{problem}{Problem}
\newtheorem{proposition}{Proposition}[section]
\newtheorem{lemma}{Lemma}[section]
\newtheorem{corollary}[theorem]{Corollary}
\newtheorem{example}{Example}[section]
\newtheorem{definition}[problem]{Definition}

\newcommand{\BEQA}{\begin{eqnarray}}
\newcommand{\EEQA}{\end{eqnarray}}
\newcommand{\define}{\stackrel{\triangle}{=}}
\bibliographystyle{IEEEtran}
\raggedbottom
\setlength{\parindent}{0pt}
\providecommand{\mbf}{\mathbf}
\providecommand{\pr}[1]{\ensuremath{\Pr\left(#1\right)}}
\providecommand{\qfunc}[1]{\ensuremath{Q\left(#1\right)}}
\providecommand{\sbrak}[1]{\ensuremath{{}\left[#1\right]}}
\providecommand{\lsbrak}[1]{\ensuremath{{}\left[#1\right.}}
\providecommand{\rsbrak}[1]{\ensuremath{{}\left.#1\right]}}
\providecommand{\brak}[1]{\ensuremath{\left(#1\right)}}
\providecommand{\lbrak}[1]{\ensuremath{\left(#1\right.}}
\providecommand{\rbrak}[1]{\ensuremath{\left.#1\right)}}
\providecommand{\cbrak}[1]{\ensuremath{\left\{#1\right\}}}
\providecommand{\lcbrak}[1]{\ensuremath{\left\{#1\right.}}
\providecommand{\rcbrak}[1]{\ensuremath{\left.#1\right\}}}
\theoremstyle{remark}
\newtheorem{rem}{Remark}
\newcommand{\sgn}{\mathop{\mathrm{sgn}}}
\providecommand{\abs}[1]{\vert#1\vert}
\providecommand{\res}[1]{\Res\displaylimits_{#1}} 
\providecommand{\norm}[1]{\lVert#1\rVert}
%\providecommand{\norm}[1]{\lVert#1\rVert}
\providecommand{\mtx}[1]{\mathbf{#1}}
\providecommand{\mean}[1]{E[ #1 ]}
\providecommand{\fourier}{\overset{\mathcal{F}}{ \rightleftharpoons}}
%\providecommand{\hilbert}{\overset{\mathcal{H}}{ \rightleftharpoons}}
\providecommand{\system}{\overset{\mathcal{H}}{ \longleftrightarrow}}
	%\newcommand{\solution}[2]{\textbf{Solution:}{#1}}
\newcommand{\solution}{\noindent \textbf{Solution: }}
\newcommand{\cosec}{\,\text{cosec}\,}
\providecommand{\dec}[2]{\ensuremath{\overset{#1}{\underset{#2}{\gtrless}}}}
\newcommand{\myvec}[1]{\ensuremath{\begin{pmatrix}#1\end{pmatrix}}}
\newcommand{\mydet}[1]{\ensuremath{\begin{vmatrix}#1\end{vmatrix}}}
\numberwithin{equation}{subsection}
\makeatletter
\@addtoreset{figure}{problem}
\makeatother
\let\StandardTheFigure\thefigure
\let\vec\mathbf
\renewcommand{\thefigure}{\theproblem}
\def\putbox#1#2#3{\makebox[0in][l]{\makebox[#1][l]{}\raisebox{\baselineskip}[0in][0in]{\raisebox{#2}[0in][0in]{#3}}}}
     \def\rightbox#1{\makebox[0in][r]{#1}}
     \def\centbox#1{\makebox[0in]{#1}}
     \def\topbox#1{\raisebox{-\baselineskip}[0in][0in]{#1}}
     \def\midbox#1{\raisebox{-0.5\baselineskip}[0in][0in]{#1}}
\vspace{3cm}
\title{AI1103 : Assignment 2}
\author{Prashanth Sriram S - CS20BTECH11039}
\maketitle
\newpage
\bigskip
\renewcommand{\thefigure}{\theenumi}
\renewcommand{\thetable}{\theenumi}
Download all python codes from 
\begin{lstlisting}
https://github.com/prashanthsriram-s/AI1103/tree/main/Assignment2/codes/
\end{lstlisting}
%
and latex codes from 
%
\begin{lstlisting}
https://github.com/prashanthsriram-s/AI1103/tree/main/Assignment2/Assignment2.tex
\end{lstlisting}
\section*{\textbf{Problem statement(GATE 67)}}
Let X and Y be random variables having the joining probability density function
\begin{align}
f\brak{x,y}=
\begin{cases}
\frac{1}{\sqrt{2\pi y}}e^{\frac{-1}{2y}\brak{x-y}^2} &-\infty<x<\infty,0<y<1\\
0 &\text{otherwise}
\end{cases}
\end{align}
The Variance of the random variable X is\\
\begin{enumerate}[label=\alph*)]
\item $\frac{1}{12}$ \\
\item $\frac{1}{4}$ \\
\item $\frac{7}{12}$ \\
\item $\frac{5}{12}$ \\
\end{enumerate}
\section*{\textbf{Solution(GATE 67)}}
Variance of the random variable X is
\begin{align}
 V\brak{X} = E\brak{X^2}-\brak{E\brak{X}}^2   
\end{align}
\begin{lemma}
\begin{align}
    \int_{-\infty}^{\infty}xe^{-\frac{1}{2y}\brak{x-y}^2}\,dx = \sqrt{2\pi}y^\frac{3}{2}
\end{align}
\end{lemma}
\begin{proof}
\begin{align}
    &\int_{-\infty}^{\infty}xe^{-\frac{1}{2y}\brak{x-y}^2}\,dx\\
  =&\int_{-\infty}^{\infty}\brak{x-y}e^{-\frac{1}{2y}\brak{x-y}^2}\,dx+y\int_{-\infty}^{\infty}e^{-\frac{1}{2y}\brak{x-y}^2}\,dx\\
 =&0+\sqrt{2\pi}y^{\frac{3}{2}}\int_{-\infty}^{\infty}\frac{1}{\sqrt{y}\sqrt{2\pi}}e^{\frac{-1}{2\brak{\sqrt{y}}^2}\brak{x-y}^2}\,dx\\
 =&\sqrt{2\pi}y^\frac{3}{2}\lim_{x_0 \to -\infty}Q\brak{\frac{x_0 - y}{\sqrt{y}}}\\
  =&\sqrt{2\pi}y^\frac{3}{2} \label{eq:2}
\end{align}
\end{proof}
\begin{lemma}
\begin{align}
E\brak{X} = \frac{1}{2}
\end{align}
\end{lemma}
\begin{proof}
\begin{align}
 E\brak{X}=&\int_{0}^{1}\int_{-\infty}^{\infty}x f_{XY}\brak{x,y}\,dx\,dy\\
    =&\int_{0}^{1}\int_{-\infty}^{\infty}x\frac{1}{\sqrt{2\pi y}}e^{\frac{-1}{2y}\brak{x-y}^2}\,dx \,dy\\
     =&\int_{0}^{1}\frac{1}{\sqrt{2\pi y}}\brak{\int_{-\infty}^{\infty}xe^{-\frac{1}{2y}\brak{x-y}^2}\,dx}\,dy \label{eq:1}
\end{align}


From \eqref{eq:1} and \eqref{eq:2}, 
\begin{align}
E\brak{X}=&\int_{0}^{1}y\,dy\\
E\brak{X}=&\frac{1}{2} \label{eq:3}
\end{align}
\end{proof}
\begin{lemma}
\begin{align}
\int_{-\infty}^{\infty}x^2e^{-\frac{1}{2y}\brak{x-y}^2}\,dx=\sqrt{2\pi}y^\frac{3}{2}\brak{y+1}
\end{align}
\end{lemma} 
\begin{proof}
\begin{align}
&\int_{-\infty}^{\infty}x^2e^{-\frac{1}{2y}\brak{x-y}^2}\,dx\\
&=\brak{\sqrt{\frac{\pi}{2}}y^\frac{3}{2}\brak{y+1}\brak{1-2Q\brak{\frac{x-y}{\sqrt{y}}}} - ye^\frac{-\brak{x-y}^2}{2y}\brak{x+y}}]_{-\infty}^{\infty}\\
&=0-\sqrt{2\pi}y^\frac{3}{2}\brak{y+1}Q\brak{\frac{x-y}{\sqrt{y}}}]_{-\infty}^{\infty} - 0\\
&=\sqrt{2\pi}y^\frac{3}{2}\brak{y+1} \label{eq:5}
\end{align}
\end{proof}
\begin{lemma}
\begin{align}
E\brak{X^2} = \frac{5}{6}
\end{align}
\end{lemma}
\begin{proof}
\begin{align}
 E\brak{X^2}=&\int_{0}^{1}\int_{-\infty}^{\infty}x^2\,f_{XY}\brak{x, y}\,dx\,dy\\   
 =&\int_{0}^{1}\int_{-\infty}^{\infty}x^2\frac{1}{\sqrt{2\pi y}}e^{\frac{-1}{2y}\brak{x-y}^2}\,dx \,dy\\
 =&\int_{0}^{1}\frac{1}{\sqrt{2\pi y}}\brak{\int_{-\infty}^{\infty}x^2e^{-\frac{1}{2y}\brak{x-y}^2}\,dx}\,dy
 \label{eq:4}
\end{align}

From \eqref{eq:4} and \eqref{eq:5}, we get
\begin{align}
E\brak{X^2} = \int_{0}^{1}y\brak{y+1}dy\\
&=\brak{\frac{y^3}{3}+\frac{y^2}{2}}]_{0}^{1}\\
&=\frac{1}{3}+\frac{1}{2}=\frac{5}{6} \label{eq:6}
\end{align}
\end{proof}
From \eqref{eq:3} and \eqref{eq:6}, we get
\begin{align}
V\brak{X} = E\brak{X^2} - \brak{E\brak{X}}^2\\
     = \frac{5}{6} - \frac{1}{4}\\
     = \frac{7}{12}
\end{align}
Therefore, the answer is (C)$\frac{7}{12}$
\end{document}